\csname PSTtransformLoaded\endcsname
\let\PSTtransformLoaded\endinput
%
\ifx\PSTricksLoaded\endinput\else\input pstricks \fi
\ifx\PSTthreeDLoaded\endinput\else\input pst-3d.tex\fi
\ifx\PSTNodeLoaded\endinput\else\input pst-node.tex\fi
\ifx\PSTXKeyLoaded\endinput\else\input pst-xkey.tex\fi
%
\def\fileversion{1.0}
\def\filedate{07/09/01}
%
%% COPYRIGHT 2007 by Manuel Luque
%% This program can be redistributed and/or modified under the terms
%% of the LaTeX Project Public License Distributed from CTAN
%% archives in directory macros/latex/base/lppl.txt.
%
\message{`PST-transform' v\fileversion, \filedate\space (ML)}

\edef\PstAtCode{\the\catcode`\@} \catcode`\@=11\relax


\def\pstranslate{\pst@object{pstranslate}}
\def\pstranslate@i(#1)(#2){\begingroup\use@par\pst@makebox{\pstranslate@ii(#1)(#2)}}
\def\pstranslate@ii(#1)(#2){%
  \pst@getcoor{#1}\pst@tempA
  \pst@getcoor{#2}\pst@tempB
  \leavevmode
\pnode(!\pst@tempA /y0 ED /x0 ED
        \pst@tempB /y1 ED /x1 ED
        0 0){factice}%
   \hbox{%
  \pst@Verb{%
    {[1 0 0 1 x1 x0 sub y1 y0 sub]
     concat } \tx@TMChange }%
    \box\pst@hbox%
\pst@Verb{\tx@TMRestore}}\endgroup}

\def\psHomothetie{\pst@object{psHomothetie}}
\def\psHomothetie@i(#1)#2{\begingroup\use@par\pst@makebox{\psHomothetie@ii(#1){#2}}}
\def\psHomothetie@ii(#1)#2{%
  \pst@getcoor{#1}\pst@tempa
  \leavevmode
\pnode(! \pst@tempa /yH ED /xH ED /K #2 def 0 0){factice}%
   \hbox{%
  \pst@Verb{%
    {[K 0 0 K K xH mul neg xH add K yH mul neg yH add]
     concat } \tx@TMChange }%
    \box\pst@hbox%
\pst@Verb{\tx@TMRestore}}\endgroup}

\def\psrotate{\pst@object{psrotate}}
\def\psrotate@i(#1)#2{\begingroup\use@par\pst@makebox{\psrotate@ii(#1){#2}}}
\def\psrotate@ii(#1)#2{%
  \pst@getcoor{#1}\pst@tempa
  \pst@getangle{#2}\pst@tempb
  \leavevmode
\pnode(!\pst@tempa /yRot ED /xRot ED 0 0){factice}%
  \hbox{%
    \pst@Verb{%
      \pst@tempb dup cos /cosA ED sin /sinA ED
      /ax cosA def
      /by sinA def
      /cx sinA neg def
      /dy cosA def
      /ex xRot cosA mul neg xRot add yRot sinA mul add def
      /fy xRot sinA mul neg yRot add yRot cosA mul sub def
       { [ax by cx dy ex fy] concat } \tx@TMChange }%
    \box\pst@hbox%
  \pst@Verb{ \tx@TMRestore }}\endgroup}

\pst@addfams{pst-cis}
\define@key[psset]{pst-cis}{cisX}{% cisaillement suivant Ox
\edef\psk@CisX{#1}}
\define@key[psset]{pst-cis}{cisY}{% cisaillement suivant Ox
\edef\psk@CisY{#1}}
\psset[pst-cis]{cisY=0,cisX=45}

\def\pscisaillement{\@ifnextchar[{\pst@cisaillement}{\pst@cisaillement[]}}
\def\pst@cisaillement[#1]{\pst@killglue\pst@makebox{\pscisaillement@[#1]}}
\def\pscisaillement@[#1]{{%
\psset{#1}%
\leavevmode
\hbox{%
\pst@Verb{
    /AngleCisX \psk@CisX\space def
    /AngleCisY \psk@CisY\space def
    /tanCisX AngleCisX sin
             AngleCisX cos
             div
             def
    /tanCisY AngleCisY sin
             AngleCisY cos
             div
             def
    /ax 1 tanCisX tanCisY mul add def
    /by tanCisY def
    /cx tanCisX def
    /dy 1 def
    { [ax by cx dy 0 0] concat
    } \tx@TMChange}%
    \box\pst@hbox
\pst@Verb{\tx@TMRestore}}%
    }}

%(#1) coordonn�es d'une extr�mit� du miroir (X1,Y1)
%(#2) coordonn�es de l'autre extr�mit� du miroir (X2,Y2)
% (A)(B) si A et B sont des noeuds d�finis avec \pnode(x1,y1){A}, \pnode(x2,y2){B}
\def\SymPlan(#1)(#2){\pst@killglue\pst@makebox{\SymPlan@(#1)(#2)}}%
\def\SymPlan@(#1)(#2){{%
\pst@getcoor{#1}\pst@tempa%
\pst@getcoor{#2}\pst@tempb%
\leavevmode
\hbox{%
\pnode(!
    \pst@tempa /Y1 exch def
    /X1 exch def
    \pst@tempb /Y2 exch def
    /X2 exch def
    /DX X2 X1 sub def
    /DY Y2 Y1 sub def
    /D12 DX dup mul DY dup mul add def
    /ax 2 DX dup mul mul D12 div 1 sub def
    /by 2 DX DY mul mul D12 div def
    /cx by def
    /dy 2 DY dup mul mul D12 div 1 sub def
    /tx X1 DY dup mul mul
    Y1 DX DY mul mul sub
    2 mul
    D12 div
    def
    /ty Y1 DX dup mul mul
    X1 DX DY mul mul sub
    2 mul
    D12 div
    def 0 0){factice}%
\pst@Verb{%
    { [ax by cx dy tx ty] concat
    } \tx@TMChange}%
    \box\pst@hbox
\pst@Verb{\tx@TMRestore}}%
    }}

\catcode`\@=\PstAtCode\relax
\endinput
